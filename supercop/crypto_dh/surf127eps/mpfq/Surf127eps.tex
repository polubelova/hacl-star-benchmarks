\documentclass{article}


\usepackage{a4wide}
\usepackage[T1]{fontenc}
\usepackage[latin1]{inputenc}
\usepackage{aeguill}
\usepackage{amsmath,amssymb,amsfonts}

\newcommand{\F}{\mathbb{F}}

\title{{\tt Surf127eps}: A key-exchange cryptosystem based on a Kummer
surface}

\author{P. Gaudry, T. Houtmann and E. Thom�}

\date{Version 2. June 2007}

\begin{document}

\maketitle

The {\tt Surf127eps} cryptosystem is a key-exchange cryptosystem based on
the Kummer surface of a genus 2 hyperelliptic curve over the finite field
with $p=2^{127}-735$ elements. The implementation follows the formulae
described in \cite{GaudryTheta}.

\section{Parameters of the Kummer surface}

The Kummer surface that is used corresponds to a genus 2 curve that has
complex multiplication by
$$K=\mathbb{Q}\left(i\sqrt{5 +\sqrt{53}}\right).$$
This field has class number 4 and is non-Galois. The
CM-ideal has degree 8 and is irreducible over $\mathbb{Q}$. Consider the
prime integer
$$ p = 2^{127}-735.$$
This prime splits completely in $K$ and the corresponding factors are
good Weil numbers. Furthermore the Igusa invariants of one of the
corresponding curves have quotient of squares of Theta constants which
are $\F_p$-rational. With the terminology of \cite{GaudryTheta}, one can
take (projectively):
$$ \begin{array}{rcl}
a & = & 1 \\
b & = & 104737609498996807573042644460938049128\\
c & = & 149790188288476750369734729103725046598\\
d & = & 129048219867477581895670028479253045173\, .
\end{array}$$
This data verifies the {\em Genericity Conditions} of \cite{GaudryTheta};
also the underlying curve is known, so that there is no need to check the
{\em Rationality Conditions}.

The CM theory provides the group orders for the Jacobians of the curve
and its twists:
$$\begin{array}{rcl}
N & = & 16\times
1809251394333065553537167681402254284155645350006293173560781593047272845093\\
\tilde{N} & = & 256 \times 2551 \times 62039 \times 167974189
\times 36109087046045143171\\
          &   & \times\ 117799819428280417184102738315899959101\, .
\end{array}$$

The following point $P$ on the Kummer surface has order $N/8$ and can
therefore be used as a base point in a Diffie-Hellman key exchange.
$$ P = (1, 1, 7, 90405191680719851590637208281830435685).$$

\noindent {\bf Security.}
We consider here the elementary security of this cryptosystem, without
taking into account the potential weaknesses of the implementation (there
are some in this version) or in the inclusion in a larger protocol
(authentication is not done, for instance). Let us then consider the best
known algorithm for solving one discrete logarithm in the $N$-order
Jacobian corresponding to this Kummer surface, namely Pollard's Rho
algorithm or distributed variants of it. This takes about the square root
of the largest prime factor of the group order: $\sqrt{N/16} \approx
2^{125}$. This is well beyond any feasible computation.

\section{Some implementation details}

\subsection{Finite field arithmetic}

Elements of the finite field are represented as a table of {\tt unsigned
long} of fixed length (2 or 4 words, depending on 32- or 64-
architecture). In the current implementation, after each operation the
returned element is normalized to an integer between $0$ and $p-1$. This
is probably not optimal. The reduction modulo $p$ takes advantage of the
particular form of $p$. Still, the assembly for AMD64 has been optimized,
and the timings are much better than in version 1.

\subsection{Encoding of keys}

A secret key is an integer between $0$ and $2^{256}-1$. We impose it to
be a multiple of $2$ in order to avoid subgroups attacks. This secret key
is stored in big-endian form in $32$ bytes. 

The public key and the shared secret are points of the Kummer surface.
Since these are projective coordinates, we can (or we must in the case of
the share secret) make them affine by, for instance, putting the first
coordinate to $1$. There are 3 elements left to store, that all fill in
$128$ bits (actually 127), therefore it takes $48$ bytes to store a public
key or a shared secret. 

\section{Future work}

Here is our "to do" list for the next version, that should be much closer to
a decent production software. This should not deteriorate the speed of the
algorithm, except maybe for the key validation part.

\begin{itemize}
\item Different representation of finite field elements.

As said above, the current representation of field element is maybe
too strict to allow maximal performance. We might change our choice of
the base field in order to have {\em nails} (GMP language).

\item Key validation.

It is necessary to check that the public key of the other party is valid.
One hope would be to avoid this completely by using an appropriate
surface that corresponds to secure curve and twist. However this is not
enough, since there is also a need to check that the point is indeed on
the Kummer surface.

\item Study and avoid degenerate cases.

The current version does not take into account the potential exceptional
cases that would create a non-point with all-zero coordinates. There is
some theory to do here before fixing the implementation.

\item Study the point compression.

There is some redundency in the $48$ bytes used for representing a Kummer
point. We can put all the information in $32$ bytes, but then it is
required to compute the roots of a polynomial of degree 4. We might
implement that in the next version.

\item Implement Montgomery's PRAC algorithm.

The Lucas chain we use is the classical double and add which is not
optimal. If we assume that resistance to side-channel attack is not an
issue, we can use the PRAC algorithm to save some work.

\item Use a random or a RM curve, obtained by point counting.

Point counting on this size of genus 2 curve has not yet been done. It is
not clear that this will be feasible in a near future without algorithmic
improvements.

One possibility is to use a curve with real multiplication. Indeed, the
point counting can take advantage of that particularity, and
cryptographical sizes are easily in reach with current technology and
algorithms. This approach is completely compatible with forcing small
Kummer parameters in order to have several multiplications by small constant
integers. Hence some speed-up is expected compared to the current system
that uses a CM curve.


\end{itemize}

\begin{thebibliography}{10}
\bibitem{GaudryTheta}
P.~Gaudry.
\newblock Fast genus 2 arithmetic based on theta functions.
\newblock Cryptology ePrint Archive: Report 2005/314, 2005.
\end{thebibliography}



\end{document}
